\conclusion
\label{conclusion}

\section{Результаты}
В результате данной работы была изучена предметная область, придуман и реализован алгоритм классификации записей из микроблогов. Реализованный алгоритм имеет несколько особенностей: он использует контекст записей, а также ``Википедию'', как источник дополнительных знаний.

Данный алгоритм был протестирован на нескольких вариантов тестовых данных. В некоторых случаях алгоритм показал хорошие результаты и продемонстрировал существенное улучшение в сравнении с простым подходом для классификации записей.

\section{Дальнейшая работа}
Большим пространством для дальнейших исследований представляется улучшение алгоритма выделения признаков из текста с использованием ``Википедии''. Стоит отметить, что возможно стоит попробовать другие алгоритмы кластеризации, например, k-medoids\footnote{http://en.wikipedia.org/wiki/K-medoids}, как не требующий представления документов в пространстве признаков (данный алгоритм требует только расстояние между элементами выборки). Имеет смысл попробовать применить латентно-семантический анализ, для предобработки признаковых представлений объектов, для улучшения результата в задачах классификации и кластеризации. Также стоит подготовить б\'oльшие обучающие и тестовые выборки для более полного тестирования полученных алгоритмов. 

Кроме того разработанный алгоритм работает очень долго, что сильно осложняет его тестирование. Было проведено профилирование, в ходе которого было обнаружено, что наибольшую часть времени занимает процесс построения кластеров и непосредственно классификации. В частности, проблема заключается в том, что код библиотеки Weka является однопоточным. Можно, к примеру, распараллелить алгоритмы кластеризации, контекстной классификации, а также тестирование методом перекрёстной проверки. 
