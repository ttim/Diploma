\intro

\section{Мотивация}
Сервисы микроблогов, такие как, например, ``Твиттер''\footnote{https://twitter.com}, являются одной из тенденций последних лет. Эти сервисы позволяют своим пользователям публиковать записи различного рода, но при этом небольшой длины (в случае сервиса ``Твиттер'' --- 140 символов). На момент апреля 2012 года сервисом пользовалось более 140 миллионов активных пользователей, которые публиковали до 1 миллиарда записей каждые 3 дня \cite{web:twitter-users}. 

Во время президентских выборов в США в 2008 году ``Твиттер'' использовался, в частности, для анализа позиций кандидатов \cite{Diakopoulos:2010:CDP:1753326.1753504}. Были попытки прогнозирования кассовых сборов фильмов в США с помощью ``Твиттера'' \cite{DBLP:journals/corr/abs-1003-5699}. 

Надо заметить, что активные пользователи сервиса могут получать сотни записей каждый день и далеко не всё могут прочитать сразу. Очень часто возникает желание сгруппировать полученные записи в зависимости от их тематики, чтобы  правильно расставить приоритеты чтения, что приводит нас к проблеме фильтрации записей.

С учетом вышесказанного становится понятно, что задача классификации сообщений из ``Твиттера'' является актуальной и востребованной. При наличии такого механизма классификации мы смогли бы решать разнообразные задачи, такие как: нахождение спама, разделение сообщений по разным тематикам, отделение тематических сообщений от бессодержательных. 

\section{Постановка задачи}
Целью данной дипломной работы является создание классификатора записей из микроблогов (а конкретно ``Твиттера''), который на основе обучающей выборки будет способен классифицировать записи. Для достижения данной цели были выделены следующие этапы: 
\begin{enumerate}
    \item изучение предметной области и существующих методов классификации записей из микроблогов;
    \item реализация алгоритма классификации использующего контекст записей;
    \item использование ``Википедии'' для получения неявных знаний о тексте;
    \item тестирование созданного классификатора.
\end{enumerate}

\section{Структура работы}
В главе \ref{chap:preliminaries} дается описание предметной области, вводятся понятия классификации, классификатора и кластеризации. Кроме того в данной главе рассказывается про основные метрики для оценки качества алгоритмов классификации и кластеризации, рассматриваются веб-сервисы ``Твиттер'' и ``Википедия''. В завершении главы описывается одна из простых моделей текста для задач классификации и кластеризации. 

В главе \ref{chap:bib} рассказывается о нынешнем состоянии исследований в области классификации и кластеризации микроблогов, в том числе и с учетом ``Википедии''.

В главе \ref{chap:main} рассказывается об алгоритме классификации, который предлагается в данной работе, приводится общая схема такого классификатора, его конкретные реализации с использованием стандартной модели текста, а также модели с использованием ``Википедии''.

В главе \ref{chap:experiments} приводится описание экспериментов, проделанных с разными вариантами классификаторов, и анализируется их результат.

В главе \ref{chap:tech} дается краткое описание технических решений использованных при создании классификатора.

В части ``\nameref{conclusion}'' подытоживаются полученные результаты и приводятся направления для дальнейших исследований.