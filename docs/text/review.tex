\chapter{Обзор литературы}
\label{chap:bib}
На текущий момент существует множество исследований связанных с вопросами классификации и кластеризации, как с использованием ``Википедии'', так и в контексте микроблогов. Существует работы классификации записей из `Твиттера` по их тональности (sentiment analysis). К примеру, в работе \cite{Go_Bhayani_Huang_2009} сообщения классифицируются по своей эмоциональной тональности; в качестве алгоритмов классификации используются метод опорных векторов, наивный байесовский метод, метод максимальной энтропии. В качестве признаков текста используются n-граммы и части речи.

Стоит отметить работу \cite{ramage2010characterizing}. В ней авторы сделали попытку кластеризовать записи из ``Твиттера'' с помощью алгоритма LDA\footnote{http://en.wikipedia.org/wiki/Latent\_Dirichlet\_allocation}. Полученные кластеры были разделены на 5 категорий: substance, status, style, social и other. 
 
Во многих работах сделаны попытки улучшения кластеризации с помощью ``Википедии''. Так в работе \cite{Genc:2011:DCC:2021773.2021833} использован сравнительно простой подход: для каждого сообщения в выборке находится наиболее релевантная статья Википедии, далее вводится метрика схожести двух статей на основе расстояния между категориями этих статей. % наверное тут еще что-то я хотел дописать

В работах \cite{Gabrilovich:2009:WSI:1622716.1622728} и \cite{Milne:2008:LLW:1458082.1458150} соответственно с помощью ``Википедии'' решаются задачи определения смысловой схожести слов, выделения ключевых слов.

Конкретно вопрос классификации записей из микроблогов рассматривается в работе \cite{Sriram:2010:STC:1835449.1835643, Horn_2010}. В первой работе рассматривается задача классификации на классы новостей, событий, мнений, предложений и личных сообщений. Во второй работе делается попытка классифицировать записи на два вида классов: на новостные, личные и от компаний, а также на мнения и факты. Также данная работа примечательна большим количеством алгоритмов участвующих в тестировании. 
