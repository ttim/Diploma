%\abstract
%Сервисы микроблогов, такие как, например, ``Твиттер''\footnote{https://twitter.com}, являются одной из тенденций последних лет. Эти сервисы позволяют своим пользователям публиковать сообщения различного рода, но при этом небольшой длины (в случае сервиса ``Твиттер'' --- 140 символов). Традиционные методы классификации текстов, основанные, к примеру, на модели bag-of-words \cite{wiki:bag-of-words}, для коротких текстов показывают себя несостоятельными. В данной работе описывается алгоритм классификации сообщений из ``Твиттера'' использующий веб-сервис ``Википедия''\footnote{http://en.wikipedia.org}, как источник дополнительных знаний, а также учитывающий другие сообщения автора классифицируемого сообщения. Также в данной работе происходит сравнение стандартного подхода классификации основанного на модели bag-of-words и предложенного алгоритма.